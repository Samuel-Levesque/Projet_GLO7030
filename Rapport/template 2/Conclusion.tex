Pour conclure, nous avons réalisé le but premier de ce projet qui était de réaliser un projet \emph{end-to-end} en traitement d'images par la création d'une interface graphique de reconnaissance de dessins manuscrits. 
Nous avons pu nous familiariser avec les outils d'infonuagique pour l'entraînement de notre réseau. 
Nous avons également un gain de performance non négligeable de 2\% avec notre méthode d'échnatillonnage ciblé pour l'entraînement de notre modèle \emph{ResNet18}.

Pour continuer d'améliorer les performances de notre réseau, plusieurs avenues peuvent être testées.
Entre autres, on pourrait modifier notre interface graphique pour extraire les composantes temporelles de nos dessins et les utiliser dans notre modèle prédictif.

Aussi, on pourrait intégrer d'autres modèles moins corrélés dans notre modèle par ensemble pour profiter des forces de chacun des classificateurs.
Notamment, on pourrait introduire des classificateurs spécialisés pour les classes les plus difficiles à prédire pour aller chercher des performances supérieures là où notre modèle performe le moins bien.
