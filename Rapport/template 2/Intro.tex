Le traitement d'images est une des sphères les plus impressionnantes et applicables du domaine de l'intelligence artificielle. 
De par la grande importance de cette branche et à cause du fait que nous n'étions pas très familiers avec le traitement d'images, nous voulions utiliser ce projet pour nous initier à un problème réel de reconnaissance d'images.

Également, nous voulions nous obliger à utiliser des solutions d'infonuagique pour traiter de grandes quantités de données que nous ne pourrions pas traiter sur nos postes personnels afin de nous rapprocher de ce qui est vraiment fait en pratique dans le milieu de l'intelligence artificielle. 

Finalement, nous avions comme but de produire une interface graphique utilisable par les utilisateurs pour tester en direct les performances de notre modèle.

Pour toutes ces raisons, la création d'une interface utilisateur de reconnaissance de dessins manuscrits en utilisant la base de données \emph{Quick, Draw!} de Google nous semblait comme un projet parfait pour nous familiariser avec les éléments que nous souhaitions approfondir.
